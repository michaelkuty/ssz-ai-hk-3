\chapter{TECH}

\section{Principy počítačů (historický vývoj, předpoklady fungování, binární logika, modulace signálu).}

\section{2.       Architektura počítače (von Neumannovo a Harwardské schéma, Flynnova taxonomie, základní deska, procesor, mikroarchitektura procesoru, paměti, sběrnice, řadič, přídavné karty, ovladače).
}

\section{Paměťový systém počítače a ukládání dat (typy, principy fungování, frekvence, normy, logická a fyzická struktura disku, RAM, ROM, Cache, HDD, CD, DVD, FLASH…)}

\section{Architektura periferních zařízení (rozdělení, principy, funkce, typy, rozhraní, příklady)}

\section{Servery a pracovní stanice (rozdíly, kritéria výběru, role serverů, serverové technologie, zálohování dat včetně RAID)}

\section{Komunikační prostředky (principy komunikace, modulace signálu, rozdělení a porovnání, média, mobilní technologie)}

\section{ETHERNET (principy fungování, vývoj a topologie, přístupová metoda, síťová karta, strukturovaná kabeláž)}

\section{RM ISO/OSI, TCP/IP (popis a srovnání, funkce zásadních protokolů, IP adresy)}

\section{Internet (organizační struktura, vývoj, RFC dokumenty, domény, technické předpoklady pro připojení, hrozby)}

\section{Směrování (základní principy, směrovací protokoly, směrovací algoritmy, směrovače)}

\section{Propojování a management sítí (přenosová média, technologie pro různé vrstvy, WIFI, VPN, systémy pro vzdálený přístup, řešení založená na SNMP)}

\section{Principy operačních systémů (základní rozdělení, druhy operačních systémů, procesy, správa procesů a systémových zdrojů, uživatelská rozhraní).}

\section{Souborové systémy a logická struktura dat (principy, porovnání, příklady).}

\section{Operační systémy Windows (principy MS DOS, MS Windows, architektura, verze, funkce, rozdíly).}

\section{ Operační systémy Unix, Linux, BSD, MacOS (základní myšlenky, výhody a nevýhody, open-source, vznik a vývoj, licence, distribuce, základy ovládání - shell, rozdíly, historie a vývoj)}

\section{Serverové operační systémy (specifika serverových operačních systémů, rozdíly mezi OS pro osobní počítač a pro server, serverové služby, správa uživatelů)}
	