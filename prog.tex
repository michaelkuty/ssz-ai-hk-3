\pagenumbering{arabic}%start arabic pagination from 1 

\chapter{PROG}

\section{ Logický program - struktura, základní pojmy, datová struktura seznam, práce s databází Prologu. Hlavní odlišnosti oproti procedurálnímu programování, možnosti použití neprocedurálního programovacího jazyka.}

\section{Databáze, databázový systém. Hlavní funkce DBS. Historický vývoj DBS. Modely dat. Relační algebra: projekce, selekce, spojení. SQL.}

\section{Konceptuální modelování. E-R model a jeho grafické znázornění. Relační model. Typy vztahů mezi entitami a jejich reprezentace v relačním modelu. Vlastnosti relační tabulky. Normální formy relačního schématu.}

\section{Ontologické inženýrství: pojem ontologie v kontextu informatiky, základní stavební prvky ontologií, typy ontologií, jazyky ontologického modelování, návrhové vzory, normalizace ontologie. Odvozování nad ontologií (kontrola konzistence, klasifikace), nástroje, použití ontologií.}

\section{Sémantický web: technologie sémantického webu, metadata, RDF, RDFS, OWL, dotazování se na sémantický web (význam, jazyky), sémantický web a odvozování (význam, jazyky), aplikace sémantického webu.}

\section{Námětové mapy: standard Topic Maps a jeho součásti, základní stavební prvky námětové mapy, postup tvorby námětové mapy, implementace námětových map (prostředí, syntaxe), dotazování se na námětové mapy, odvozování s námětovými mapami, aplikace námětových map.}

\section{Objektové modelování a programování - základní pojmy, podstata, využití. Softwarový proces. UML. Událostmi řízené programování. Architektura MVC.}

\newpage

\subsection{Událostmi řízené programování - Event-driven programming}

Událost (Event) vzniká buď jako výsledek interakce GUI s uživatelem nebo jako důsledek změny vnitřního stavu aplikace či OS
Obsluhou události nazýváme úsek kódu, který je při vzniku události automaticky vyvolán a provádí činnost k události připojenou (někdy také ohlasová metoda události či Event Handler).
Příklady typů událostí:
\begin{itemize}
\item Klik/DvojKlik
\item Zaměření/Ztráta zaměření
\item Změna stavu komponenty
\item Stisk, uvolnění klávesy
\item Stisknutí, uvolnění tl. myši
\item Posun myši
\item Událost časovače
\item Zpráva systému
\end{itemize}

\section{ Práce s kolekcemi – typy kolekcí, příklady použití, algoritmy pracující nad kolekcemi (řazení, vyhledávání), základní principy implementace ve zvoleném programovacím jazyce.}

\section{Problematika perzistentního (trvalého) ukládání dat ve vybraném programovacím jazyce.}

\section{ Webové aplikace – principy, nástroje. Vícevrstvé aplikace. Zabezpečení aplikace.}

\section{Základní algoritmy a principy počítačové grafiky  – metody vizualizace, určení viditelnosti a osvětlení, reprezentace grafické informace, OPENGL.}

\section{Základy zpracování obrazu a počítačového rozpoznávání – metody snímání, předzpracování, segmentace a klasifikace obrazu, formáty pro ukládání rastrového obrazu, komprese, barva a barevné modely.}

\section{Algoritmy pracující s grafy. Prohledávání grafů do hloubky a do šířky, využití prohledávání grafů v dalších úlohách.  }